\chapter{Source Code}

Please be aware that the indentations in this code will be visible in the compiled PDF. Many styles are predefined for you, change them in \texttt{preamble/code.tex} if needed.

\section{Markup and data languages}
\subsection{\LaTeX{} code example}
\begin{lstlisting}[style=latexstyle]
\documentclass[
    oneside,
    bibliography=totocnumbered, % Include bibliography in TOC with number
    listof=totocnumbered,       % Include List of Tables/Figures in TOC with number
    numbers=noenddot            % Changes heading numbering from 1.1. to 1.1
    % appendixprefix = true     % Makes Appendix A... First appendix chapter, disable for one appendix
]{scrbook}
%%%%%%%%%%%%%%%%%%%%%%%%%%%%%%%%%%%%%%%%%%%%
\begin{document}
    Hello, world!

    \begin{equation}
        \pdv{\psi}{t}
        =
        \frac{\partial \psi}{\partial t}
    \end{equation}

\end{document}
\end{lstlisting}



\subsection{YAML code example}
You can also directly input the code of a file into the \verb|lstlisting| command.

\lstinputlisting[style=yamlstyle]{code/docker-compose.yaml}

\section{Programming languages}
\subsection{Python code example}
\begin{lstlisting}[style=pythonstyle]
def fibonacci(n):
    a, b = 0, 1
    while a < n:
        print(a, end=" ")
        a, b = b, a + b
    print()

# Example usage
fibonacci(50)

\end{lstlisting}

\subsection{Matlab code example}
\begin{lstlisting}[style=matlabstyle]
function fibonacci(n)
    a = 0;
    b = 1;
    while a < n
        fprintf('%d ', a);
        temp = a + b;
        a = b;
        b = temp;
    end
    fprintf('\n');
end

% Example usage
fibonacci(50)
\end{lstlisting}


\subsection{Wolfram Mathematica code example}
\begin{lstlisting}[style=wolframstyle]
FibonacciSequence[n_] := Module[{a = 0, b = 1, temp},
    While[a < n,
        Print[a, " "];
        temp = a + b;
        a = b;
        b = temp;
    ]
]

(* Example usage *)
FibonacciSequence[50]

\end{lstlisting}


\subsection{C code example}
\begin{lstlisting}[style=cstyle]
#include <stdio.h>

void fibonacci(int n) {
    int a = 0, b = 1, temp;
    while (a < n) {
        printf("%d ", a);
        temp = a + b;
        a = b;
        b = temp;
    }
    printf("\n");
}

int main() {
    fibonacci(50);
    return 0;
}
\end{lstlisting}




%---------------------------------------------------------------------------%
\clearpage
\subsection{Longer code example (C)}

The following is the main.cpp from the TeXstudio GitHub repository. Direct Link:

https://github.com/texstudio-org/texstudio/blob/master/src/main.cpp

\begin{lstlisting}[style=cstyle]
/***************************************************************************
*   copyright       : (C) 2003-2007 by Pascal Brachet                     *
*   addons by Frederic Devernay <frederic.devernay@m4x.org>               *
*   addons by Luis Silvestre                                              *
*   http://www.xm1math.net/texmaker/                                      *
*                                                                         *
*   This program is free software; you can redistribute it and/or modify  *
*   it under the terms of the GNU General Public License as published by  *
*   the Free Software Foundation; either version 2 of the License, or     *
*   (at your option) any later version.                                   *
*                                                                         *
***************************************************************************/

#include "mostQtHeaders.h"
/*! \mainpage TexStudio
*
* \see Texstudio
* \see PDFDocument
*/

#include "texstudio.h"
#include "smallUsefulFunctions.h"
#include "debughelper.h"
#include "debuglogger.h"
#include "utilsVersion.h"
#include <qtsingleapplication.h>
#include <QSplashScreen>

#ifdef Q_OS_WIN32
#include "windows.h"
typedef BOOL (WINAPI *AllowSetForegroundWindowFunc)(DWORD);
#endif

class TexstudioApp : public QtSingleApplication
{
	public:
	bool initialized;
	QString delayedFileLoad;
	Texstudio *mw;  // Moved from private:
	TexstudioApp(int &argc, char **argv);
	TexstudioApp(QString &id, int &argc, char **argv);
	~TexstudioApp();
	void init(QStringList &cmdLine);   // This function does all the initialization instead of the constructor.
	/*bool notify(QObject* obj, QEvent* event){
		//really slow global event logging:
		//qWarning(qPrintable(QString("%1 obj %2 named %3 typed %4 child of %5 received %6").arg(QTime::currentTime().toString("HH:mm:ss:zzz")).arg((long)obj,8,16).arg(obj->objectName()).arg(obj->metaObject()->className()).arg(obj->parent()?obj->parent()->metaObject()->className():"").arg(event->type())));
		try {
			return QApplication::notify(obj,event);
		} catch (const std::exception& e){
			qDebug() << "Catched exception: " << e.what();
			return false;
		}
	}*/


	protected:
	bool event(QEvent *event);
};

TexstudioApp::TexstudioApp(int &argc, char **argv) : QtSingleApplication(argc, argv)
{
	mw = nullptr;
	initialized = false;
}

TexstudioApp::TexstudioApp(QString &id, int &argc, char **argv) : QtSingleApplication(id, argc, argv)
{
	mw = nullptr;
	initialized = false;
}

void TexstudioApp::init(QStringList &cmdLine)
{
	QPixmap pixmap(":/images/splash.png");
	QSplashScreen *splash = new QSplashScreen(pixmap);
	splash->show();
	processEvents();

	mw = new Texstudio(nullptr, Qt::WindowFlags(), splash);
	connect(this, SIGNAL(lastWindowClosed()), this, SLOT(quit()));
	splash->finish(mw);
	delete splash;

	initialized = true;

	if (!delayedFileLoad.isEmpty()) cmdLine << delayedFileLoad;
	mw->executeCommandLine(cmdLine, true);
	if(!cmdLine.contains("--auto-tests")){
		mw->startupCompleted();
	}
}

TexstudioApp::~TexstudioApp()
{
	delete mw;
}

bool TexstudioApp::event(QEvent *event)
{
	if (event->type() == QEvent::FileOpen) {
		QFileOpenEvent *oe = static_cast<QFileOpenEvent *>(event);
		if (initialized) mw->load(oe->file());
		else delayedFileLoad = oe->file();
		event->accept();
		return true;
	}
	return QApplication::event(event);
}

QString generateAppId()
{
	QProcessEnvironment env = QProcessEnvironment::systemEnvironment();
	QString user = env.value("USER");
	if (user.isEmpty()) {
		user = env.value("USERNAME");
	}
	return QString("%1_%2").arg(TEXSTUDIO,user);
}

QStringList parseArguments(const QStringList &args, bool &outStartAlways)
{
	QStringList cmdLine;
	for (int i = 1; i < args.count(); ++i) {
		QString cmdArgument =  args[i];

		if (cmdArgument.startsWith('-')) {
			// various commands
			if (cmdArgument == "--start-always")
			outStartAlways = true;
			else if (cmdArgument == "--no-session")
			ConfigManager::dontRestoreSession = true;
			else if ((cmdArgument == "-line" || cmdArgument == "--line") && (++i < args.count()))
			cmdLine << "--line" << args[i];
			else if ((cmdArgument == "-page" || cmdArgument == "--page") && (++i < args.count()))
			cmdLine << "--page" << args[i];
			else if ((cmdArgument == "-insert-cite" || cmdArgument == "--insert-cite") && (++i < args.count()))
			cmdLine << "--insert-cite" << args[i];
			else if (cmdArgument == "--ini-file" && (++i < args.count())) {
				// deprecated: use --config instead
				ConfigManager::configDirOverride = QFileInfo(args[i]).absolutePath();
			}
			else if (cmdArgument == "--config" && (++i < args.count()))
			ConfigManager::configDirOverride = args[i];
			#ifdef DEBUG_LOGGER
			else if ((cmdArgument == "--debug-logfile") && (++i < args.count()))
			debugLoggerStart(args[i]);
			#endif
			else
			cmdLine << cmdArgument;
		} else {
			if(cmdArgument.endsWith(".txss")||cmdArgument.endsWith(".txss2")){
				// explicit session restor
				// disable restoring of last session
				ConfigManager::dontRestoreSession = true;
			}
			cmdLine << QFileInfo(cmdArgument).absoluteFilePath();
		}
	}
	return cmdLine;
}

bool handleCommandLineOnly(const QStringList &cmdLine) {
	// note: stdout is not supported for Win GUI applications. Will simply not output anything there.
	if (cmdLine.contains("--help")) {
		QTextStream(stdout) << "Usage: texstudio [options] [file]\n"
		<< "\n"
		<< "Options:\n"
		<< "  --config DIR              use the specified settings directory\n"
		<< "  --master                  define the document as explicit root document\n"
		<< "  --line LINE[:COL]         position the cursor at line LINE and column COL\n"
		<< "  --insert-cite CITATION    inserts the given citation\n"
		<< "  --start-always            start a new instance, even if TXS is already running\n"
		<< "  --pdf-viewer-only         run as a standalone pdf viewer without an editor\n"
		<< "  --page PAGENUM            display a certain page in the pdf viewer\n"
		<< "  --no-session              do not load/save the session at startup/close\n"
		<< "  --texpath PATH            force resetting command defaults with PATH as first search path\n"
		<< "  --version                 show version number\n"
		#ifdef DEBUG_LOGGER
		<< "  --debug-logfile pathname  write debug messages to pathname\n"
		#endif
		;
		return true;
	}

	if (cmdLine.contains("--version")) {
		QTextStream(stdout) << "TeXstudio " << TXSVERSION << " (" << TEXSTUDIO_GIT_REVISION << ")\n";
		return true;
	}

	return false;
}

int main(int argc, char **argv)
{
	QString appId = generateAppId();
	#if QT_VERSION >= QT_VERSION_CHECK(5,6,0)
	if(qEnvironmentVariableIntValue("TEXSTUDIO_HIDPI_SCALE")>0){
		QApplication::setAttribute(Qt::AA_EnableHighDpiScaling);
	} else {
		QApplication::setAttribute(Qt::AA_DisableHighDpiScaling);
	}
	#endif
	// This is a dummy constructor so that the programs loads fast.
	TexstudioApp a(appId, argc, argv);
	bool startAlways = false;
	QStringList cmdLine = parseArguments(QCoreApplication::arguments(), startAlways);

	if (handleCommandLineOnly(cmdLine)) {
		return 0;
	}

	if (!startAlways) {
		if (a.isRunning()) {
			#ifdef Q_OS_WIN32
			AllowSetForegroundWindowFunc asfw = (AllowSetForegroundWindowFunc) GetProcAddress(GetModuleHandleA("user32.dll"), "AllowSetForegroundWindow");
			if (asfw) asfw(/*ASFW_ANY*/(DWORD)(-1));
			#endif
			a.sendMessage(cmdLine.join("#!#"));
			return 0;
		}
	}

	a.setApplicationName( TEXSTUDIO );
	#if (QT_VERSION >= QT_VERSION_CHECK(5, 7, 0)) && defined(Q_OS_LINUX)
	a.setDesktopFileName("texstudio");
	#endif
	a.init(cmdLine); // Initialization takes place only if there is no other instance running.

	QObject::connect(&a, SIGNAL(messageReceived(const QString&)),
	a.mw, SLOT(onOtherInstanceMessage(const QString&)));

	try {
		int execResult = a.exec();
		#ifdef DEBUG_LOGGER
		if (debugLoggerIsLogging()) {
			debugLoggerStop();
		}
		#endif
		return execResult;
	} catch (...) {
		#ifndef NO_CRASH_HANDLER
		catchUnhandledException();
		#endif
		throw;
	}
}
\end{lstlisting}