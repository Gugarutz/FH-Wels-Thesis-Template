\section{Numbered and unnumbered lists}
This serves as a demonstration of the \verb|itemize| and \verb|enumerate| environments and their possibilities.

\textcolor[rgb]{0.7, 0, 0}{WARNING}: \verb|\subitem| is not used for lists, it is a command used in indices.

\textbf{Note:} The spacing of the items has been reduced compared to the default value for higher density of the text.

\subsection{Itemize}

    \begin{itemize}
        \item First level item 1
        \item First level item 2
        \item First level item 3
        \item First level item 4
        \item First level item 5
    \end{itemize}

    You can use \verb|\begin{itemize} ... \end{itemize}| within itself (nesting) to create sub items.

    \textbf{Note:} In this document, \texttt{itemize} lists use horizontal rules in place of default bullets, with size decreasing by level.

    \begin{itemize}
        \item Level 1 item
        \begin{itemize}
            \item Level 2 item
            \begin{itemize}
                \item Level 3 item
                \begin{itemize}
                    \item Level 4 item
                \end{itemize}
            \end{itemize}
        \end{itemize}
    \end{itemize}


\subsection{Enumerate}

    \begin{enumerate}
        \item First level item 1
        \item First level item 2
        \item First level item 3
        \item First level item 4
        \item First level item 5
    \end{enumerate}

    Here is an example of a nested \verb|enumerate| environment.
    \texttt{enumerate} lists have been customized to show hierarchical numbering:
    \texttt{1., 1.1., 1.1.1.,} etc.

    \begin{enumerate}
        \item First level item 1
        \begin{enumerate}
            \item Second level item 1
            \begin{enumerate}
                \item Third level item 1
                \begin{enumerate}
                    \item Fourth level item 1
                \end{enumerate}
            \end{enumerate}
        \end{enumerate}
    \end{enumerate}

%------------------------------------------------------------%
\subsection{Outline package}

This example uses the \texttt{outlines} package (note the plural). A separate and unrelated \texttt{outline} package also exists -- be sure to refer to the correct one.

The customization done via \texttt{enumitem} carry over to the outline environment.

    \begin{outline}
        \1 Item on level 1
        \2 Item on level 2
        \3 Item on level 3
        \4 Item on level 4
    \end{outline}

    Here is what that looks like in the code:

    \begin{verbatim}
    \begin{outline}
          \1 Item on level 1
          \2 Item on level 2
          \3 Item on level 3
          \4 Item on level 4
    \end{outline}
    \end{verbatim}

    The \verb|\1| etc. provide the "level" indicator for the list.

    Here is an \verb|outline| environment using the \verb|[enumerate]| option right after\\ \verb|\begin{outline}|.

    \begin{outline}[enumerate]
        \1 Top level item
         \2 Sub item
          \3 sub sub item
    \end{outline}

    You can use a one-time custom label with square brackets, e.g. \verb|\1[>]|. The default label (line or number, depending on choice) is overwritten.

    \begin{outline}
        \1[>] a very simple example using the normal greater sign
        \2[$\rightarrowtail$] you can use the math commands, here: \verb|[$\rightarrowtail$]|
        \2[\LaTeX] be creative
        \2[$\cdot$] or don't
        \3[\textit{fixed right:}] yes, all of that is the label. Be careful about the left margin!
    \end{outline}

