\section{Figures}
    \label{sec:Figures}

    Figures in \LaTeX{} are handled using the \texttt{figure} environment. Like tables, figures are inserted as floats. This means they are positioned where \LaTeX{} deems optimal for page layout, rather than exactly where they are written in the code. By default, floats tend to move to the top or bottom of pages or onto dedicated float pages.

    To force figures closer to their position in the code, use placement specifiers:
    \begin{itemize}
        \item \texttt{[h]} — roughly "here"
        \item \texttt{[H]} — exactly "here" (requires the \texttt{float} package)
    \end{itemize}

    It is best practice to scale images relative to the text width, e.g., \verb|width=0.5\textwidth|, to maintain consistency across different document sizes.

    \begin{figure}[h]
    	\centering
    	\includegraphics[width=0.50\textwidth]{FH.png}
    	\caption{Example of inserting a figure}
    	\label{fig:ExampleFigure-1}
    \end{figure}

%-------------------------------------------------------------------------
\subsection{Multiple figures in one environment}

    The simplest way to display two images side by side is to call \verb|\includegraphics| twice inside one \texttt{figure} environment. However, spacing must be managed manually.

    Example with manual spacing:

    \begin{figure}[h]
        \centering
        \includegraphics[width=0.4\textwidth]{FH.png}
        \hfil
        \includegraphics[width=0.4\textwidth]{FH.png}
        \caption{Example of inserting two figures}
        \label{fig:ExampleFigure-2}
    \end{figure}

    \verb|\hfil| provides flexible horizontal spacing and is generally the best choice.

    Alternatives include \verb|\hspace|, \verb|\hskip|, and explicit spacing lengths.

    \paragraph{Using \texttt{subfigure} (Recommended)}

    A more robust method uses the \texttt{subcaption} package. This allows each image to have its own caption and label for referencing.

    \begin{figure}[H]
        \centering
        \begin{subfigure}{0.4\textwidth}
            \centering
            \includegraphics[width=\textwidth]{FH.png}
            \caption{First Image}
            \label{fig:subfig1}
        \end{subfigure}
        \hfil
        \begin{subfigure}{0.55\textwidth}
            \centering
            \includegraphics[width=\textwidth]{FH.png}
            \caption{Second Image}
            \label{fig:subfig2}
        \end{subfigure}
        \caption{Example of inserting a figure with differently sized subfigures with subcaption}
        \label{fig:ExampleFigure-4}
    \end{figure}

    \begin{warningbox}
        A \texttt{subfigure} package also exists. It is incompatible with \texttt{subcaption} and is less versatile.
    \end{warningbox}

    The widths of subfigures should sum to less than or equal to 1.0 relative to \verb|\textwidth| or \verb|\linewidth|, accounting for spacing. Typical choices are \verb|0.49\linewidth| or \verb|0.33\linewidth|.

    \begin{notebox}
        There is a difference between \verb|\textwidth| and \verb|\linewidth|.

        \verb|\textwidth| refers to the total width of the main text block on the page. It stays constant in single-column documents.

        \verb|\linewidth| refers to the width of the current line of text — which changes depending on the environment. Inside a figure, table, or other floats, \verb|\linewidth| is equal to \verb|\textwidth|.

        However, inside nested environments like minipage, subfigure, quote, or wrapfigure, \verb|\linewidth| adjusts to the local width.

        This means that when sizing images inside subfigures or wrapped figures, you should generally use width=\verb|\linewidth| rather than \verb|\textwidth| to ensure the image fills the container correctly.
    \end{notebox}

%-------------------------------------------------------------------------
\subsection{Usage of the overpic package}

    The overpic package is too big to be explained here. Please refer to its own documentation. You can use the grid option to help with positioning.
    \begin{figure}[H]
        \centering\vskip5mm
        \begin{overpic}[width=0.5\textwidth,grid,tics=20]{FH.png}
            \put(40,35){\color{red} \Large Example Annotation}
        \end{overpic}
        \caption{Example using the overpic package}
        \label{fig:overpic_example}
    \end{figure}


%-------------------------------------------------------------------------
    \clearpage
\subsection{Usage of wrapfig package}

%\setlength{\intextsep}{0pt}   % Removes space above & below the figure
%\setlength{\columnsep}{5pt}   % Adjusts space between figure and text


\begin{wrapfigure}{r}{0.4\textwidth}  % 'r' for right placement
    \vspace{-10pt}  % Reduce space above the figure
    \centering
    \includegraphics[width=\linewidth]{FH.png}
    \caption{A wrapped figure with reduced spacing.}
    \label{fig:wrapfig}
\end{wrapfigure}



This is an example of a paragraph with a wrapped figure. The text will flow around the image automatically. You can adjust the figure width, placement, and spacing to optimize the layout. If the text is too short, you might see issues with overlapping or strange formatting.
This is an example of a paragraph with a wrapped figure. The text will flow around the image automatically. You can adjust the figure width, placement, and spacing to optimize the layout. If the text is too short, you might see issues with overlapping or strange formatting.
This is an example of a paragraph with a wrapped figure. The text will flow around the image automatically. You can adjust the figure width, placement, and spacing to optimize the layout. If the text is too short, you might see issues with overlapping or strange formatting.

The spacing of the image has been adjusted to move it further upwards.

\subsection{On using figures}
Consider the following points when using figures:
\begin{itemize}
    \item Figures must be referenced within or linked to the text; otherwise they are superfluous. To do this, use the command \verb|\cref{}|
    [see \cref{fig:ExampleFigure-1}]\\
    An additional page reference can be added with the command \verb|\pageref{}| \newline
    [see \cref{fig:ExampleFigure-1} on page \pageref{fig:ExampleFigure-1}]
    \item Not every figure needs to be in the text. Analyses, tables, backup information, etc.~can be collected in the appendix. However, the text should reference them.
    \item If necessary, add a legend in the figures and the necessary references.
\end{itemize}
