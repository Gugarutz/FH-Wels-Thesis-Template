\section{Insert a table}

    This section demonstrates how to create tables using different environments and styling options.

    Tables in \LaTeX{} are highly customizable but can be verbose. The examples below show basic layouts, row colouring, automatic column width, and data import from external files.

%------------------------------------------------------------%
\paragraph{Simple Table with Rulings}
    This example shows a standard table with manual horizontal and vertical lines. Different ruling styles (|, \verb|\hline|, \verb|\headrule|) help visually separate headers and rows. This is the most basic way to create tables.

    Note: Manual layout like this is error-prone for complex tables or when column widths need to adapt.

    \begin{table}[H]
    	\centering
    	\caption{Simple Table with Rulings}
    	\begin{tabular}{c|c||c}
    		Heading column 1 & Heading column 2 & Heading column 3 \\ \headrule
    		       1         &      Test-1      &      Test-2      \\ \hline
    		       2         &        0         &        0         \\
    		       3         &        0         &        0         \\
    		       4         &        0         &        1         \\
    	\end{tabular}
    	\label{tab:table-rulings}
    \end{table}

%------------------------------------------------------------%
\paragraph{Table with Alternating Row Colors}
    Here, the \verb|\rowcolors| command is used to improve readability by applying alternating row colors (zebra striping). This technique is particularly helpful for tables with many rows.

    However, the columns use fixed-width definitions (p\{3cm\}), which allows text wrapping but must be set manually.

    \begin{table}[H]
        \centering
        \caption{Table with alternating row colours}
        \rowcolors{2}{gray!20}{white}  % Alternating row colors (light gray and white)
        \begin{tabular}{|p{3cm}|p{3cm}|p{3cm}|}
            \hline
            size     & size 1   & size 2   \\ \hline
            [unit]   & [unit 1] & [unit 2] \\ \hline
            object 1 & 5        & 6        \\
            object 2 & 8        & 9        \\
            object 3 & 11       & 12       \\ \hline
        \end{tabular}
    \end{table}

%------------------------------------------------------------%
\paragraph{Table with Automatic Column Width — Tabularx}
    The tabularx package automatically adjusts column widths to fit the table into a specified width (here, 80\% of the text width).
    This reduces manual tweaking but is less compatible with features like \verb|\rowcolors| or certain advanced table rules.

    Columns labelled X auto-expand.

    Mixed with standard c or fixed-width p columns.

    \begin{table}[H]
        \centering
        \caption{Table with alternating colours and automatic column width (B) using tabularx, fixed size (C) and standard centred (A)}
        \rowcolors{2}{white}{gray!15}
        \begin{tabularx}{0.8\linewidth}{c|XX p{2cm}}
            \hline
            A  & B  & B  & C  \\ \hline
            1  & 2  & 2  & 3  \\
            4  & 5  & 5  & 6  \\
            7  & 8  & 8  & 9  \\
            10 & 11 & 11 & 12 \\ \hline
        \end{tabularx}
    \end{table}

\paragraph{Table with Justified Columns — Tabulary}
    The \texttt{tabulary} package extends \texttt{tabularx} by offering precise control over text alignment within columns. It supports the following alignment options:

    \begin{itemize}
        \item \textbf{L} $\rightarrowtail$ Left aligned (equivalent to X)
        \item \textbf{C} $\rightarrowtail$ Centred
        \item \textbf{R} $\rightarrowtail$ Right aligned
        \item \textbf{J} $\rightarrowtail$ Fully justified text
    \end{itemize}

    This is particularly useful for tables that combine descriptive text with numerical data, allowing for readable layouts without manual width adjustments.

    \begin{notebox}
        Compatibility with \verb|\rowcolors| and very wide tables is limited. Be cautious when mixing with automatic row coloring or complex layouts.
    \end{notebox}

    \begin{table}[H]
        \centering
        \caption{Example using tabulary, smaller text to save space}
        \small
        \begin{tabulary}{\linewidth}{|L|C|R|J|}
            \hline
            Left-aligned & Centred & Right-aligned & Justified text                                    \\ \hline
            Apple        &    10    &          5.99 & This is some longer text that will wrap properly. \\
            Banana       &    20    &          3.49 & Another line with text to test justification.     \\
            Cherry       &    15    &          7.99 & Yet another example.                              \\ \hline
        \end{tabulary}
    \end{table}

%------------------------------------------------------------%
\section{Longtable and importing data}
%------------------------------------------------------------%

    This section shows how to create tables that automatically split across multiple pages and how to import data from external files like CSVs.

    \begin{tipbox}
        See also \cref{sec:pgf-table} for a different solution to data importing.
    \end{tipbox}


\subsection{Table from an Imported CSV (DTL)}
    This table is generated directly from a CSV file using the \text{datatool} package. This avoids manual data entry in LaTeX and ensures consistency if the data updates.

    Useful for:
    \begin{itemize}
        \item Simple data tables
        \item Automatically generated or exported datasets (e.g., from Excel)
    \end{itemize}

    \DTLloaddb{data2}{data/sample_data.csv}
    \begin{table}[H]
        \centering
        \caption{Table from an imported CSV}
        \begin{tabular}{l*{3}{r}}
            \toprule
            \textbf{Col1} & \textbf{Col2} & \textbf{Col3} & \textbf{ColPct} \\
            \DTLforeach*{data2}{\colone=Col1, \coltwo=Col2, \colthree=Col3, \colfour=ColPct}
            {%
                \DTLiffirstrow{\\ \midrule}{\\}%
                \colone & \coltwo & \colthree & \colfour %
            }
            \\ \hline
        \end{tabular}
    \end{table}


\subsection{Longtable example}
    When tables are too long to fit on one page, \texttt{longtable} handles automatic splitting with repeated headers. Combined with \texttt{datatool}, this creates dynamic, multi-page tables.

    This example loads a CSV file \texttt{(sales\_data.csv)} and displays the data with:
    \begin{itemize}
        \item Repeated headers on each page
        \item Alternating row colors
        \item Custom column widths
    \end{itemize}
    This method is robust but comes with more complex syntax.



    \begin{tipbox}
        External data import works best for datasets that might change frequently or for long reports.

        It can be useful to place the code for these imports in a separate file for better readability and reusability
    \end{tipbox}
    \begin{quirkbox}
        \texttt{longtable} does not work inside floating environments like \texttt{table}.
    \end{quirkbox}


\DTLloaddb{data}{data/sales_data.csv}
\begingroup
    \small
    \setlength{\tabcolsep}{3mm}
    \rowcolors{2}{gray!15}{white}
\begin{longtable}{l*{4}{r}}
    \caption{Longtable from a CSV file}\cr
    \hline
    \textbf{Date} & \textbf{Product} & \textbf{Units Sold} & \textbf{Price (€)} & \textbf{Total (€)} \\
    \midrule
    \endfirsthead

    \multicolumn{5}{c}{\textit{Continued from previous page(opt.)}} \\ \hline
    \textbf{Date} & \textbf{Product} & \textbf{Units Sold} & \textbf{Price (€)} & \textbf{Total (€)} \\
    \midrule
    \endhead

    \hline
    \multicolumn{5}{r}{\textit{(Continued on next page)}} \\
    \hline
    \endfoot

    \hline
    \endlastfoot

    % Data from your CSV file
    \DTLforeach*{data}{\datecol=Date, \productcol=Product, \quantitycol=Quantity Sold, \pricecol=Price (€), \totalcol=Total (€)}{%
        \DTLiffirstrow{}{\\} % No extra line before the first row
        \datecol & \productcol & \quantitycol & \pricecol & \totalcol %
    }
    \label{tab:multicsv}
\end{longtable}
\endgroup