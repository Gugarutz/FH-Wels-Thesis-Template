\section{Mathematics and Units}
\begingroup \small

Writing mathematical equation in \LaTeX{} is generally easy, the problem lies in making them look correct. There are many common practices that are specific to \TeX{} users, scientific writing in general, and can be very different depending of the field of study. This template focuses on the usage for mechanical \& electrical engineering for central Europe.

\subsection{Variables and Constants}
    \label{sec:variables-const}

    It is common practice to write variables in \textit{italics} and constants upright.
    For example, $k$ may be the stiffness of a spring but would be typeset as $\mathrm{k}$ (though more commonly $\mathrm{k_B}$) for the Boltzmann constant.

    Keep in mind that the usage of these principles also applies to Greek letters for example: $\upalpha$ and $\alpha$, or $\epsilon$ and $\varepsilon$. Be aware of $\pi$ and $\varpi$ especially, \verb|\varpi| (the second one) looks more like a lower case omega, than pi.

    \begin{notebox}
        Euler's constant \e was formatted in \verb|\mathscr{}| instead of upright (\verb|\mathrm{}|) for better usability in text and aesthetic purposes
    \end{notebox}

    \begin{tipbox}
        For more examples, please see the following repositories:\\
        \url{https://github.com/Gugarutz/FH-Wels-Formelsammlung-THD-STL-WUE}\\
        \url{https://github.com/Gugarutz/FH-Wels-FEL-Formelsammlung}
    \end{tipbox}

\subsection{Mathematical fonts}
    The use of fonts (or rather typefaces) in \LaTeX{} can be hard to understand, especially due to the misuse of the word by Microsoft Office products. Fonts are variations of a single typeface, eg. bold, italics, or small caps. Whereas typefaces are what you may know as fonts.

    Keep in mind that the typefaces for math and text are always separated in \LaTeX{}, both in regards of setting the defaults and in terms of availability.

    \begin{center}
        \small
        \begin{tabular}{lccl}
            \toprule
            \textbf{Command}        & \textbf{Text}          & \textbf{Numbers}           & \textbf{Description}                     \\ \midrule
            Default (none)          & $ABC$                  & $123456789.0$              & Default math italic                      \\
            \verb|\mathrm{...}|     & $\mathrm{ABC-abc}$     & $\mathrm{123456789.0}$     & Upright Roman (for constants)                    \\
            \verb|\mathit{...}|     & $\mathit{ABC-abc}$     & $\mathit{123456789.0}$     & Italic (for variables)         \\
            \verb|\mathbf{...}|     & $\mathbf{ABC-abc}$     & $\mathbf{123456789.0}$     & Bold Roman (serif)                       \\
            \verb|\mathsf{...}|     & $\mathsf{ABC-abc}$     & $\mathsf{123456789.0}$     & Sans-serif upright                       \\
            \verb|\mathtt{...}|     & $\mathtt{ABC-abc}$     & $\mathtt{123456789.0}$     & Typewriter (monospace)                   \\
            \verb|\mathcal{...}|    & $\mathcal{ABC}$        & ---                        & Calligraphic (uppercase only)            \\
            \verb|\mathbb{...}|     & $\mathbb{ABC}$         & ---                        & Blackboard (uppercase only)              \\
            \verb|\mathds{...}|     & $\mathds{ABC}$         & $\mathds{123456789.0}$     & alternate Blackboard (uppercase only)    \\
            \verb|\mathfrak{...}|   & $\mathfrak{ABC-abc}$   & $\mathfrak{123456789.0}$   & Fraktur (Gothic-style)                   \\
            \verb|\boldsymbol{...}| & $\boldsymbol{ABC-abc}$ & $\boldsymbol{123456789.0}$ & General bold math (symbols \& Greek) \\
            \verb|\mathscr{...}|    & $\mathscr{ABC}$        & ---                        & Script (more decorative)    \\ \bottomrule
        \end{tabular}
    \end{center}



\subsection{Sub and superscript}
    Normal subscript and superscript comes after the object in question and is done with~\texttt{\_} and~\texttt{\^}. By default this only applies to one symbol each, so \verb|M_ab| produces $M_ab$. To avoid this use curly brackets like with other \LaTeX{} commands as follows: \verb|M_{ab}^{cd}| to produce $M_{ab}^{cd}$. Also avoid $M_{a^b}$ (\verb*|M_{a^b}|) and ${M_a}^b$ (\verb*|{M_a}^b|).

    With the \verb|\prescript{a}{b}{M}| command you can typeset subscript and superscript before an object like this:
    \[
        \prescript{a}{b}{M}
        \text{\quad and even add the ones after that: \quad}
        \prescript{a}{b}{M}_c^d
    \]



\subsection{Physics package}

    The \verb|physics| package defines \verb|\qty|. However this is already used by \texttt{siunitx}. Therefore this template redefines \verb|\qty| to \verb|\set|.

    \begin{center}
        \verb|\set{10}|             produces: $\set{10}$

        \verb|\set[\dfrac{13}{14}]| produces: $\set[\dfrac{13}{14}]$

        \verb/\set|\vec{a}| \abs{\vec{r}} \norm{\va{r}} \norm{\va*{r}}/ \\
            produces: \\
            $\set|\vec{a}| \quad  \abs{\vec{r}} \quad \norm{\va{r}} \quad \norm{\va*{r}} $
    \end{center}


    \begin{warningbox}
        \verb*|\set| may only be used inside a math environment!
    \end{warningbox}

    \subsubsection{Differentials}

        The \texttt{physics} packages adds various helpful commands for easy and quick setting of differentials.

        The normal upright d used in by \texttt{physics} was replaced by $\mathrm{d\!I}$ which is used by professor Schiefermayr in his scripts. You can disable this in the \texttt{preamble/math.tex} under the custom math commands.

         \[ \dd[3]{x} \]
         \[ \int_{0}^{\infty} 3x^2 \dd{x} = \eval{x^3}_{0}^{\infty} \]

\subsection{Greek Symbols in Text}

    With the \verb|textgreek| package, you can use commands like \verb|\textalpha| (\textalpha) to create Greek characters inside normal text without needing to invoke math mode.

%==================================================================================
\subsection{The \texttt{siunitx} Package}

    % The `siunitx` package ensures correct typesetting of numbers and units.
    This package is useful for automatic typesetting of numbers and units. You will need to choose appropriate settings in \verb|\unitsetup{...}| in the \verb|main.tex| preamble. % TODO: Specify the exact settings path.

    \paragraph*{Only Units}
        The \verb|\unit| command ensures proper spacing and styling for standalone units:
        \begin{itemize}
            \item \verb|\unit{\kilogram\metre\per\second}| produces \unit{\kilogram\metre\per\second}.
            \item \verb|\unit{\kg\per\m\per\s\squared}| produces \unit{\kg\per\m\per\s\squared}.
            \item \verb|\unit{\degreeCelsius}| correctly handles special symbols: \unit{\degreeCelsius}.
            \item \verb|\unit{\ohm}| ensures correct typesetting of units like ohms: \unit{\ohm}.
        \end{itemize}

    \paragraph*{Only Numbers}
        The \verb|\num| command formats numbers consistently:
        \begin{itemize}
            \item \verb|\num{12345.678}| formats large numbers: \num{12345.678}.
            \item \verb|\num{6.022e23}| correctly formats scientific notation: \num{6.022e23}.
            \item \verb|\num{000.000123}| removes unnecessary leading zeros: \num{000.000123}.
        \end{itemize}

    \paragraph*{Numbers and Units}
        The \verb|\qty| command combines numbers and units seamlessly:
        \begin{itemize}
            \item \verb|\qty{9.81}{\metre\per\second\squared}| represents acceleration: \qty{9.81}{\metre\per\second\squared}.
            \item \verb|\qty{1.5}{\mega\hertz}| correctly applies SI prefixes: \qty{1.5}{\mega\hertz}.
            \item \verb|\qty{500}{\milli\liter}| handles subunits properly: \qty{500}{\milli\liter}.
        \end{itemize}

    \paragraph*{Uncertainty}
        \begin{itemize}
            \item \verb|\num{1.23(45)}| shows uncertainty in parentheses: \num{1.23(45)}.
            \item \verb|\qty{1.23(45)}{\metre}| combines uncertainty with units: \qty{1.23(45)}{\metre}.
            \item \verb|\num{1.23(4)e5}| formats numbers with uncertainty in scientific notation: \num{1.23(4)e5}.
        \end{itemize}
        To display uncertainty separately, use:
        \begin{itemize}
            \item \verb|\qty[separate-uncertainty=false]{1.23(0.05)}{\metre}|, \\
            which produces \qty[separate-uncertainty=false]{1.23(0.05)}{\metre}.
        \end{itemize}


    \paragraph*{Display Options}
        Different display options for units and numeric notation can be configured in \\
         \verb|\unitsetup{...}| globally or locally.

        The \verb|\num| command can be forced into scientific notation using the\\
        \verb|scientific-notation=true| option:

        \begin{itemize}
            \item \verb|\num[exponent-mode=scientific]{000.000123}| produces \num[exponent-mode=scientific]{000.000123}.
        \end{itemize}

        The \verb|\unit| and \verb|\qty| command can format units as fractions:
        \begin{itemize}
            \item \verb|\unit[per-mode=fraction]{\kg\per\m\per\s\squared}|
               produces \unit[per-mode=fraction]{\kg\per\m\per\s\squared}.
            \item \verb|\unit[per-mode=symbol]{\kg\per\m\per\s\squared}|
               produces \unit[per-mode=symbol]{\kg\per\m\per\s\squared}.
        \end{itemize}

%==================================================================================
\subsection{Custom commands}
    Some special mathematical characters are provided by this template in a nicer and distinct font or typeset upright as per \cref{sec:variables-const}.

    \begin{align*}
        \text{Euler's constant: } & \texttt{\tb e    } \rightarrowtail  \e    \\
        \text{Imaginary i: }      & \texttt{\tb imag } \rightarrowtail  \imag \\
        \text{Imaginary j: }      & \texttt{\tb jmag } \rightarrowtail  \jmag
    \end{align*}

    \subsubsection{Number spaces}
        The following commands are provided number sets:
        \[
            \verb|\N| \rightarrowtail \N \quad
            \verb|\Z| \rightarrowtail \Z \quad
            \verb|\Q| \rightarrowtail \Q \quad
            \verb|\R| \rightarrowtail \R \quad
            \verb|\I| \rightarrowtail \I \quad
            \verb|\C| \rightarrowtail \C
        \]

    \subsubsection{Umlauts in math mode}
        Some extra math shortcuts are provided for German users for Umlauts as the normal ä, ö, ü would be typeset upright and require brackets \{ \} despite being one letter.

        \[
            \verb|\aum| \rightarrowtail  \aum \qquad
            \verb|\oum| \rightarrowtail  \oum \qquad
            \verb|\uum| \rightarrowtail  \uum
        \]
        \[
            \verb|\Aum| \rightarrowtail  \Aum \qquad
            \verb|\Oum| \rightarrowtail  \Oum \qquad
            \verb|\Uum| \rightarrowtail  \Uum
        \]

        Compared to normal Umlauts being used in math mode:
        % faked to not have error: 'Command \" not allowed in math mode'
        \[
            P_{\ddot{\mathrm{a}}} \rightarrowtail P_\aum \qquad
            T_{\ddot{\mathrm{o}}} \rightarrowtail T_\oum \qquad
            p_{\ddot{\mathrm{u}}} \rightarrowtail p_\uum
        \]
        \[
            P_{\ddot{\mathrm{A}}} \rightarrowtail P_\Aum \qquad
            T_{\ddot{\mathrm{O}}} \rightarrowtail T_\Oum \qquad
            p_{\ddot{\mathrm{U}}} \rightarrowtail p_\Uum
        \]


    \subsubsection{Dimensionless numbers}
        This Package provides a significant set of preset named, dimensionless numbers. Be aware that these are not simple shortcuts but have specific kerning (the distance between characters) for better readability and recognizability.

        The commands can be found in \texttt{preamble/set-math.tex} file, you may disable them there. See table below for all available shortcuts/commands.

        \begin{table}[H]
            \centering
            \caption{All dimensionless numbers provided}
            \begin{tabular}{lll|lll|lll|lll}
                \verb|\Arch| & $\rightarrowtail $ & $\Arch$ &
                \verb|\Lewi| & $\rightarrowtail $ & $\Lewi$ &
                \verb|\Four| & $\rightarrowtail $ & $\Four$ &
                \verb|\Rayl| & $\rightarrowtail $ & $\Rayl$ \\
                \verb|\Biot| & $\rightarrowtail $ & $\Biot$ &
                \verb|\Mach| & $\rightarrowtail $ & $\Mach$ &
                \verb|\Frou| & $\rightarrowtail $ & $\Frou$ &
                \verb|\Reyn| & $\rightarrowtail $ & $\Reyn$ \\
                \verb|\Cauc| & $\rightarrowtail $ & $\Cauc$ &
                \verb|\Nuss| & $\rightarrowtail $ & $\Nuss$ &
                \verb|\Gras| & $\rightarrowtail $ & $\Gras$ &
                \verb|\Schm| & $\rightarrowtail $ & $\Schm$ \\
                \verb|\Damk| & $\rightarrowtail $ & $\Damk$ &
                \verb|\Pecl| & $\rightarrowtail $ & $\Pecl$ &
                \verb|\Karl| & $\rightarrowtail $ & $\Karl$ &
                \verb|\Sher| & $\rightarrowtail $ & $\Sher$ \\
                \verb|\Eule| & $\rightarrowtail $ & $\Eule$ &
                \verb|\Pran| & $\rightarrowtail $ & $\Pran$ &
                \verb|\Knud| & $\rightarrowtail $ & $\Knud$ &
                \verb|\Stro| & $\rightarrowtail $ & $\Stro$ \\
                \verb|\Webe| & $\rightarrowtail $ & $\Webe$ &
                &&&&&&&&
            \end{tabular}
        \end{table}



    \subsubsection{Abbreviations}
        \begin{center}
        \begin{tabular}{lll|lll|lll|lll}
            \verb|\eff|   & $\rightarrowtail$ & $\eff$   &
            \verb|\avg|   & $\rightarrowtail$ & $\avg$   &
            \verb|\maxx|  & $\rightarrowtail$ & $\maxx$  &
            \verb|\minn|  & $\rightarrowtail$ & $\minn$  \\
            \verb|\inl|   & $\rightarrowtail$ & $\inl$   &
            \verb|\out|   & $\rightarrowtail$ & $\out$   &
            \verb|\refe|  & $\rightarrowtail$ & $\refe$  &
            \verb|\sat|   & $\rightarrowtail$ & $\sat$   \\
            \verb|\amb|   & $\rightarrowtail$ & $\amb$   &
            \verb|\crit|  & $\rightarrowtail$ & $\crit$  &
            \verb|\env|   & $\rightarrowtail$ & $\env$   &
            \verb|\tot|   & $\rightarrowtail$ & $\tot$   \\
            \verb|\res|   & $\rightarrowtail$ & $\res$   &
            \verb|\num|   & $\rightarrowtail$ & $\numer$   &
            \verb|\expmt| & $\rightarrowtail$ & $\expmt$ &
            \verb|\simu|  & $\rightarrowtail$ & $\simu$  \\
            \verb|\init|  & $\rightarrowtail$ & $\init$  &
            \verb|\final| & $\rightarrowtail$ & $\final$ &
            \verb|\thrm|  & $\rightarrowtail$ & $\thrm$  &
            \verb|\cond|  & $\rightarrowtail$ & $\cond$  \\
            \verb|\conv|  & $\rightarrowtail$ & $\conv$  &
            \verb|\rad|   & $\rightarrowtail$ & $\rad$   &
            \verb|\gen|   & $\rightarrowtail$ & $\gen$   &
            \verb|\loss|  & $\rightarrowtail$ & $\loss$  \\
            \verb|\trans| & $\rightarrowtail$ & $\trans$ &
            \verb|\eq|    & $\rightarrowtail$ & $\eq$    &
            \verb|\rxn|   & $\rightarrowtail$ & $\rxn$   &
            \verb|\mix|   & $\rightarrowtail$ & $\mix$   \\
            \verb|\bulk|  & $\rightarrowtail$ & $\bulk$  &
            \verb|\wall|  & $\rightarrowtail$ & $\wall$  &
            \verb|\film|  & $\rightarrowtail$ & $\film$  &
            \verb|\surf|  & $\rightarrowtail$ & $\surf$  \\
            \verb|\rms|   & $\rightarrowtail$ & $\rms$   &
            \verb|\dc|    & $\rightarrowtail$ & $\dc$    &
            \verb|\ac|    & $\rightarrowtail$ & $\ac$    &
            \verb|\nom|   & $\rightarrowtail$ & $\nom$   \\
            \verb|\ctrl|  & $\rightarrowtail$ & $\ctrl$  &
            \verb|\meas|  & $\rightarrowtail$ & $\meas$  &
            \verb|\set|   & $\rightarrowtail$ & $\setp$   &
            &&
        \end{tabular}
        \end{center}

\endgroup