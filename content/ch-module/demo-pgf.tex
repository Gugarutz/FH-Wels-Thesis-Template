
\section{Using PGF for tables}
\label{sec:pgf-table}

PGF (Portable Graphics Format) and its higher-level interface TikZ can be used to create tables in LaTeX. However, PGF/TikZ is primarily designed for graphics and diagrams, so it's not the most common method for making tables.

The \verb|pgfplotstable| packages makes it possible to create tables with PGF. It uses CSV syntax and does not understand strings (letters and words) by default, keep that in mind.

It is not meant to be used for short tables, but more for CSV files and large amount of data.

\begin{table}[H]
    \centering\caption{Example tables utilizing PGF}
    \pgfplotstabletypeset[
    col sep=comma, % Use comma as column separator
    header=true,   % Display the first row as a header
    columns/Name/.style={string type}, % Treat the "Name" column as text
    ]{Name, Age, Score
        Alice, 23, 88
        Bob, 25, 92
        Charlie, 22, 85
    }
    \hskip 1cm
    \pgfplotstabletypeset[
    col sep=comma, % Use comma as column separator
    header=true,   % Display header row
    columns/Name/.style={string type}, % Treat "Name" column as text
    every head row/.style={before row=\toprule, after row=\hline\hline}, % Top and mid rules
    every last row/.style={after row=\bottomrule}, % Bottom rule
    columns={Name,Age,Score}, % Specify column order
    every row no 2/.style={before row=\midrule} % Insert midrule before 3rd row
    ]{Name, Age, Score
        Alice, 23, 88
        Bob, 25, 92
        Charlie, 22, 85
    }
    \hskip 1cm
    \pgfplotstabletypeset[
    col sep=comma,
    header=true,
    columns/Name/.style={string type, column type=|l|}, % Vertical line after "Name"
    columns/Age/.style={column type=c|}, % Vertical line after "Age"
    columns/Score/.style={column type=r|}, % No vertical line after "Score"
    every head row/.style={before row=\toprule, after row=\midrule},
    every last row/.style={after row=\bottomrule}
    ]{Name, Age, Score
        Alice, 23, 88
        Bob, 25, 92
        Charlie, 22, 85
    }
\end{table}

Here you also see the different rules provided by \texttt{booktabs}. Notice that the vertical rules break the horizontal lines. This can be avoided by using custom commands provided. E.g. \verb|\topruled| instead of \verb|\toprule|, the same goes for \verb|\midruled| and \verb|\bottomruled|.