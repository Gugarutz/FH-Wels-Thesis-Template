
\usepackage{mathtools}  % Extension package to amsmath incl. fixes for bugs inn amsmath, loads 'amsmath'
\usepackage{amssymb}    % Mathematical symbols from American Math Society
\usepackage{thmtools}   % extension package to amsthm, loads 'amsthm'


\usepackage{icomma}		% disable spacing when using commas as decimal seperator

%------------------------------------------------------------%
% Math Fonts
%------------------------------------------------------------%

\usepackage{dsfont}     % \mathds
\usepackage{newtxmath}	% nicer math font that enables \mathscr


%------------------------------------------------------------%
% Avoid conflict siunitx and physics (both define \qty)
%------------------------------------------------------------%
	\usepackage{physics}
    \NewCommandCopy\set\qty
	\usepackage{siunitx}

% defines \set with the physics definition
    \AtBeginDocument{\RenewCommandCopy\set\qty}
% defines \qty as \SI (they are identical within siunitx)
    \AtBeginDocument{\RenewCommandCopy\qty\SI}

% silences the warning that \qty is overwritten see:
% https://tex.stackexchange.com/questions/681700/silencing-siunitx-and-physics-package-qty-warning
    \ExplSyntaxOn
    \msg_redirect_name:nnn { siunitx } { physics-pkg } { none }
    \ExplSyntaxOff

%------------------------------------------------------------%
% CUSTOM MATH COMMANDS
%------------------------------------------------------------%

% CHANGES TO EXISTING COMMANDS
    % replace normal differential d from physics with Schiefermayr d
    \renewcommand{\diffd}{\mathrm{d\!I}}


\newcommand*{\imag}{\ensuremath{\mathscr{i}}}
\newcommand*{\jmag}{\ensuremath{\mathscr{j}}}
\newcommand*{\e}   {\ensuremath{\mathscr{e}}}

%
\newcommand*{\N}{\ensuremath{\mathds{N}}}
\newcommand*{\Z}{\ensuremath{\mathds{Z}}}
\newcommand*{\Q}{\ensuremath{\mathds{Q}}}
\newcommand*{\R}{\ensuremath{\mathds{R}}}
\newcommand*{\I}{\ensuremath{\mathds{I}}}
\newcommand*{\C}{\ensuremath{\mathds{C}}}

% UMLAUTS
    % standard umlaut are upright in math mode
    \newcommand*{\uum}{{\ensuremath{\ddot{u}}}}
    \newcommand*{\oum}{{\ensuremath{\ddot{o}}}}
    \newcommand*{\aum}{{\ensuremath{\ddot{a}}}}
    % Capitals
    \newcommand*{\Uum}{{\ensuremath{\ddot{U}}}}
    \newcommand*{\Oum}{{\ensuremath{\ddot{O}}}}
    \newcommand*{\Aum}{{\ensuremath{\ddot{A}}}}

% DIMENSIONLESS NUMBERS
    \newcommand*{\Arch}{{\ensuremath{A\kern-.06em r}}} % http://de.wikipedia.org/wiki/Archimedes-Zahl
    \newcommand*{\Biot}{{\ensuremath{B\kern-.08em i}}} % http://de.wikipedia.org/wiki/Biot-Zahl
    \newcommand*{\Cauc}{{\ensuremath{C\kern-.13em a}}} % http://de.wikipedia.org/wiki/Cauchy-Zahl
    \newcommand*{\Damk}{{\ensuremath{D\kern-.12em a}}} % http://de.wikipedia.org/wiki/Damk%C3%B6hler-Zahl
    \newcommand*{\Eule}{{\ensuremath{E\kern-.09em u}}} % http://de.wikipedia.org/wiki/Euler-Zahl
    \newcommand*{\Four}{{\ensuremath{F\kern-.20em o}}} % http://de.wikipedia.org/wiki/Fourier-Zahl
    \newcommand*{\Frou}{{\ensuremath{F\kern-.07em r}}} % http://de.wikipedia.org/wiki/Froude-Zahl
    \newcommand*{\Gras}{{\ensuremath{G\kern-.12em r}}} % http://de.wikipedia.org/wiki/Grashof-Zahl
    \newcommand*{\Karl}{{\ensuremath{K\kern-.15em a}}} % http://de.wikipedia.org/wiki/Karlovitz-Zahl
    \newcommand*{\Knud}{{\ensuremath{K\kern-.13em n}}} % http://de.wikipedia.org/wiki/Knudsen-Zahl
    \newcommand*{\Lewi}{{\ensuremath{L\kern-.05em e}}} % http://de.wikipedia.org/wiki/Lewis-Zahl
    \newcommand*{\Mach}{{\ensuremath{M\kern-.20em a}}} % http://de.wikipedia.org/wiki/Mach-Zahl
    \newcommand*{\Nuss}{{\ensuremath{N\kern-.15em u}}} % http://de.wikipedia.org/wiki/Nusselt-Zahl
    \newcommand*{\Pecl}{{\ensuremath{P\kern-.15em e}}} % http://de.wikipedia.org/wiki/P%C3%A9clet-Zahl
    \newcommand*{\Pran}{{\ensuremath{P\kern-.10em r}}} % http://de.wikipedia.org/wiki/Prandtl-Zahl
    \newcommand*{\Rayl}{{\ensuremath{R\kern-.10em a}}} % http://de.wikipedia.org/wiki/Rayleigh-Zahl
    \newcommand*{\Reyn}{{\ensuremath{R\kern-.10em e}}} % http://de.wikipedia.org/wiki/Reynolds-Zahl
    \newcommand*{\Schm}{{\ensuremath{S\kern-.12em c}}} % http://de.wikipedia.org/wiki/Schmidt-Zahl
    \newcommand*{\Sher}{{\ensuremath{S\kern-.12em h}}} % http://de.wikipedia.org/wiki/Sherwood-Zahl
    \newcommand*{\Stro}{{\ensuremath{S\kern-.09em r}}} % http://de.wikipedia.org/wiki/Strouhal-Zahl
    \newcommand*{\Webe}{{\ensuremath{W\kern-.25em e}}} % http://de.wikipedia.org/wiki/Weber-Zahl



% KERNING
    \newcommand*{\eff}{{\kern-.1em e \kern-.20em f \kern-.30em f}}


% for readability
    \newcommand{\tb}{\textbackslash}