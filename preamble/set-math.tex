
\usepackage{mathtools}  % Extension package to amsmath incl. fixes for bugs inn amsmath, loads 'amsmath'
\usepackage{amssymb}    % Mathematical symbols from American Math Society
\usepackage{thmtools}   % extension package to amsthm, loads 'amsthm'


\usepackage{icomma}		% disable spacing when using commas as decimal seperator

%------------------------------------------------------------%
% Math Fonts
%------------------------------------------------------------%

\usepackage{dsfont}     % \mathds
\usepackage{newtxmath}	% nicer math font that enables \mathscr


%------------------------------------------------------------%
% Avoid conflict siunitx and physics (both define \qty)
%------------------------------------------------------------%
	\usepackage{physics}
    \NewCommandCopy\set\qty
	\usepackage{siunitx}

% defines \set with the physics definition
    \AtBeginDocument{\RenewCommandCopy\set\qty}
% defines \qty as \SI (they are identical within siunitx)
    \AtBeginDocument{\RenewCommandCopy\qty\SI}

% silences the warning that \qty is overwritten see:
% https://tex.stackexchange.com/questions/681700/silencing-siunitx-and-physics-package-qty-warning
    \ExplSyntaxOn
    \msg_redirect_name:nnn { siunitx } { physics-pkg } { none }
    \ExplSyntaxOff

%------------------------------------------------------------%
% CUSTOM MATH COMMANDS
%------------------------------------------------------------%

% changes to existing commands
\renewcommand{\diffd}{\mathrm{d\!I}} % replace normal differential d from physics with Schiefermayr d


\newcommand{\I}{\ensuremath{\mathds{I}}}



% DIMENSIONLESS NUMBERS
    \newcommand{\Arch}{A\kern-.06em r} % http://de.wikipedia.org/wiki/Archimedes-Zahl
    \newcommand{\Biot}{B\kern-.08em i} % http://de.wikipedia.org/wiki/Biot-Zahl
    \newcommand{\Cauc}{C\kern-.13em a} % http://de.wikipedia.org/wiki/Cauchy-Zahl
    \newcommand{\Damk}{D\kern-.12em a} % http://de.wikipedia.org/wiki/Damk%C3%B6hler-Zahl
    \newcommand{\Eule}{E\kern-.09em u} % http://de.wikipedia.org/wiki/Euler-Zahl
    \newcommand{\Four}{F\kern-.20em o} % http://de.wikipedia.org/wiki/Fourier-Zahl
    \newcommand{\Frou}{F\kern-.07em r} % http://de.wikipedia.org/wiki/Froude-Zahl
    \newcommand{\Gras}{G\kern-.12em r} % http://de.wikipedia.org/wiki/Grashof-Zahl
    \newcommand{\Karl}{K\kern-.15em a} % http://de.wikipedia.org/wiki/Karlovitz-Zahl
    \newcommand{\Knud}{K\kern-.13em n} % http://de.wikipedia.org/wiki/Knudsen-Zahl
    \newcommand{\Lewi}{L\kern-.05em e} % http://de.wikipedia.org/wiki/Lewis-Zahl
    \newcommand{\Mach}{M\kern-.20em a} % http://de.wikipedia.org/wiki/Mach-Zahl
    \newcommand{\Nuss}{N\kern-.15em u} % http://de.wikipedia.org/wiki/Nusselt-Zahl
    \newcommand{\Pecl}{P\kern-.15em e} % http://de.wikipedia.org/wiki/P%C3%A9clet-Zahl
    \newcommand{\Pran}{P\kern-.10em r} % http://de.wikipedia.org/wiki/Prandtl-Zahl
    \newcommand{\Rayl}{R\kern-.10em a} % http://de.wikipedia.org/wiki/Rayleigh-Zahl
    \newcommand{\Reyn}{R\kern-.10em e} % http://de.wikipedia.org/wiki/Reynolds-Zahl
    \newcommand{\Schm}{S\kern-.12em c} % http://de.wikipedia.org/wiki/Schmidt-Zahl
    \newcommand{\Sher}{S\kern-.12em h} % http://de.wikipedia.org/wiki/Sherwood-Zahl
    \newcommand{\Stro}{S\kern-.09em r} % http://de.wikipedia.org/wiki/Strouhal-Zahl
    \newcommand{\Webe}{W\kern-.25em e} % http://de.wikipedia.org/wiki/Weber-Zahl

% extra commands
\newcommand{\uum}{\ddot{u}} % einfaches ü nicht verwenden, da nicht kursiv
\newcommand{\oum}{\ddot{o}}
\newcommand{\aum}{\ddot{a}}
\newcommand{\eff}{\kern-.1em e \kern-.20em f \kern-.30em f}



\newcommand{\tb}{\textbackslash}