
\usepackage{mathtools}  % Extension package to amsmath incl. fixes for bugs inn amsmath, loads 'amsmath'
\usepackage{amssymb}    % Mathematical symbols from American Math Society
\usepackage{thmtools}   % extension package to amsthm, loads 'amsthm'


\usepackage{icomma}		% disable spacing when using commas as decimal seperator

%------------------------------------------------------------%
% Math Fonts
%------------------------------------------------------------%

\usepackage{dsfont}     % \mathds
\usepackage{newtxmath}	% nicer math font that enables \mathscr


%------------------------------------------------------------%
% Avoid conflict siunitx and physics (both define \qty)
%------------------------------------------------------------%
	\usepackage{physics}
    \NewCommandCopy\set\qty
	\usepackage{siunitx}

% defines \set with the physics definition
    \AtBeginDocument{\RenewCommandCopy\set\qty}
% defines \qty as \SI (they are identical within siunitx)
    \AtBeginDocument{\RenewCommandCopy\qty\SI}

% silences the warning that \qty is overwritten see:
% https://tex.stackexchange.com/questions/681700/silencing-siunitx-and-physics-package-qty-warning
    \ExplSyntaxOn
    \msg_redirect_name:nnn { siunitx } { physics-pkg } { none }
    \ExplSyntaxOff

%------------------------------------------------------------%
% CUSTOM MATH COMMANDS
%------------------------------------------------------------%

% changes to existing commands
\renewcommand{\diffd}{\mathrm{d\!I}} % replace normal differential d from physics with Schiefermayr d


\newcommand{\I}{\ensuremath{\mathds{I}}}



% DIMENSIONLESS NUMBERS
    \newcommand{\Arch}{\operatorname{\mathit{A\kern-.06em r}}} % http://de.wikipedia.org/wiki/Archimedes-Zahl
    \newcommand{\Biot}{\operatorname{\mathit{B\kern-.10em i}}} % http://de.wikipedia.org/wiki/Biot-Zahl
    \newcommand{\Cauc}{\operatorname{\mathit{C\kern-.10em a}}} % http://de.wikipedia.org/wiki/Cauchy-Zahl
    \newcommand{\Damk}{\operatorname{\mathit{D\kern-.06em a}}} % http://de.wikipedia.org/wiki/Damk%C3%B6hler-Zahl
    \newcommand{\Eule}{\operatorname{\mathit{E\kern-.03em u}}} % http://de.wikipedia.org/wiki/Euler-Zahl
    \newcommand{\Four}{\operatorname{\mathit{F\kern-.20em o}}} % http://de.wikipedia.org/wiki/Fourier-Zahl
    \newcommand{\Frou}{\operatorname{\mathit{F\kern-.07em r}}} % http://de.wikipedia.org/wiki/Froude-Zahl
    \newcommand{\Gras}{\operatorname{\mathit{G\kern-.05em r}}} % http://de.wikipedia.org/wiki/Grashof-Zahl
    \newcommand{\Karl}{\operatorname{\mathit{K\kern-.15em a}}} % http://de.wikipedia.org/wiki/Karlovitz-Zahl
    \newcommand{\Knud}{\operatorname{\mathit{K\kern-.11em n}}} % http://de.wikipedia.org/wiki/Knudsen-Zahl
    \newcommand{\Lewi}{\operatorname{\mathit{L\kern-.05em e}}} % http://de.wikipedia.org/wiki/Lewis-Zahl
    \newcommand{\Mach}{\operatorname{\mathit{M\kern-.20em a}}} % http://de.wikipedia.org/wiki/Mach-Zahl
    \newcommand{\Nuss}{\operatorname{\mathit{N\kern-.20em u}}} % http://de.wikipedia.org/wiki/Nusselt-Zahl
    \newcommand{\Pecl}{\operatorname{\mathit{P\kern-.08em e}}} % http://de.wikipedia.org/wiki/P%C3%A9clet-Zahl
    \newcommand{\Pran}{\operatorname{\mathit{P\kern-.10em r}}} % http://de.wikipedia.org/wiki/Prandtl-Zahl
    \newcommand{\Rayl}{\operatorname{\mathit{R\kern-.10em a}}} % http://de.wikipedia.org/wiki/Rayleigh-Zahl
    \newcommand{\Reyn}{\operatorname{\mathit{R\kern-.10em e}}} % http://de.wikipedia.org/wiki/Reynolds-Zahl
    \newcommand{\Schm}{\operatorname{\mathit{S\kern-.07em c}}} % http://de.wikipedia.org/wiki/Schmidt-Zahl
    \newcommand{\Sher}{\operatorname{\mathit{S\kern-.07em h}}} % http://de.wikipedia.org/wiki/Sherwood-Zahl
    \newcommand{\Stro}{\operatorname{\mathit{S\kern-.07em r}}} % http://de.wikipedia.org/wiki/Strouhal-Zahl
    \newcommand{\Webe}{\operatorname{\mathit{W\kern-.30em e}}} % http://de.wikipedia.org/wiki/Weber-Zahl

% extra commands
\newcommand{\uum}{\ddot{u}} % einfaches ü nicht verwenden, da nicht kursiv
\newcommand{\oum}{\ddot{o}}
\newcommand{\aum}{\ddot{a}}
\newcommand{\eff}{\kern-.1em e \kern-.20em f \kern-.30em f}



\newcommand{\tb}{\textbackslash}