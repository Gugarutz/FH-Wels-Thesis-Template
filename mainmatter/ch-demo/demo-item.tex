\section{Numbered and unnumbered lists}
This serves as a demonstration of the \verb|itemize| and \verb|enumerate| environments and their possibilities.

\textcolor[rgb]{0.7, 0, 0}{WARNING}: \verb|\subitem| is not used for lists, it is a command used in indices.

\subsection{Itemize}

    \begin{itemize}
        \item First level item 1
        \item First level item 2
        \item First level item 3
        \item First level item 4
        \item First level item 5
    \end{itemize}

    You can use \verb|\begin{itemize} ... \end{itemize}}| within itself (nesting) to create sub items.

    \begin{itemize}
        \item First level item 1
        \begin{itemize}
            \item Second level item 1
            \begin{itemize}
                \item Third level item 1
                \begin{itemize}
                    \item Fourth level item 1
                \end{itemize}
            \end{itemize}
        \end{itemize}
        \item First level item 2
        \item First level item 3
        \item First level item 4
    \end{itemize}

\subsection{Enumerate}

    \begin{enumerate}
        \item First level item 1
        \item First level item 2
        \item First level item 3
        \item First level item 4
        \item First level item 5
    \end{enumerate}

    Here is an example of a nested \verb|enumerate| environment.

    \begin{enumerate}
        \item First level item 1
        \begin{enumerate}
            \item Second level item 1
            \begin{enumerate}
                \item Third level item 1
                \begin{enumerate}
                    \item Fourth level item 1
                \end{enumerate}
            \end{enumerate}
        \end{enumerate}
        \item First level item 2
        \item First level item 3
        \item First level item 4
        \item First level item 5
    \end{enumerate}

%------------------------------------------------------------%
\subsection{Outline package}

Here is a standard usage of the \verb|outline| environment. It is confusingly provided by the outline\underline{s} package. An \texttt{outline} package also exists.

    \begin{outline}
        \1 Item on level 1
        \2 Item on level 2
        \3 Item on level 3
        \4 Item on level 4
    \end{outline}

    Here is what that looks like in the code:

    \begin{verbatim}
    \begin{outline}
          \1 Item on level 1
          \2 Item on level 2
          \3 Item on level 3
          \4 Item on level 4
    \end{outline}
    \end{verbatim}

    The \verb|\1| etc. provide the "level" indicator for the list.

    Here is an \verb|outline| environment using the \verb|[enumerate]| option right after\\ \verb|\begin{outline}|.

    \begin{outline}[enumerate]
        \1 Top level item
         \2 Sub item
          \3 sub sub item
    \end{outline}

    You can use one time custom label with square brackets, e.g. \verb|\1[>]|.

    \begin{outline}
        \1[>] a very simple example using the normal greater sign
        \2[$\rightarrowtail$] you can use the math commands \verb|[$\rightarrowtail$]|
        \2[\LaTeX] be creative
        \2[$\cdot$] or don't
        \3[\textit{fixed right:}] yes, all of that is label. Be careful about the left margin!
    \end{outline}

