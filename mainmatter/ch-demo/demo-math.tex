\section{Mathematics and units}

\subsection{Greek symbols in text}

    With the \verb|textgreek| package you can use commands like \verb|\textalpha| (\textalpha) to use create Greek characters inside normal text, with needing to invoke math mode.

\subsection{Variables \& constants}

	It is common practice to write variables in \textit{italics} and constants upright. For example $e$ might be used for eccentricity and $\mathrm{e} $ is used for Euler's constant $(=\num{2.718281}\dots)$.

	% todo
	todo: table with all math shortcuts


\subsection{The \texttt{siunitx} package}

	This package is useful for automatic typesetting of numbers and/or units. You will need to choose the some setting in % todo path of \sisetup{...}

	\paragraph*{Only units}
	The \verb|\si| command is used for unit formatting. It ensures proper spacing and styling for standalone units:  \\
	- \verb|\si{\kilogram\metre\per\second}| produces \si{\kilogram\metre\per\second}.  \\
	- \verb|\si{\kg\per\m\per\s\squared}| produces \si{\kg\per\m\per\s\squared}.  \\
	- \verb|\si{\degreeCelsius}| correctly handles special symbols: \si{\degreeCelsius}.  \\
	- \verb|\si{\ohm}| ensures correct typesetting of units like ohms: \si{\ohm}.

	\paragraph*{Only numbers}
	The \verb|\num| command formats numbers consistently, handling spacing, digit grouping, and scientific notation:  \\
	- \verb|\num{12345.678}| formats large numbers: \num{12345.678}. \\
	- \verb|\num{6.022e23}| correctly formats scientific notation: \num{6.022e23}.  \\
	- \verb|\num{000.000123}| removes leading zeros where necessary: \num{000.000123}.

	\paragraph*{Numbers and units}
	The \verb|\SI| command combines numbers and units seamlessly:  \\
	- \verb|\SI{9.81}{\metre\per\second\squared}| represents acceleration:\\ \SI{9.81}{\metre\per\second\squared}.  \\
	- \verb|\SI{1.5}{\mega\hertz}| correctly applies SI prefixes: \SI{1.5}{\mega\hertz}.  \\
	- \verb|\SI{500}{\milli\liter}| handles subunits properly: \SI{500}{\milli\liter}.

	\paragraph*{Uncertainty}
	- The \verb|\num{1.23(45)}| command shows uncertainty in parentheses, producing \num{1.23(45)}. \\
	- The \verb|\SI{1.23(45)}{\metre}| command combines uncertainty with units, producing \SI{1.23(45)}{\metre}.  \\
	- The \verb|\num{1.23(4)e5}| command formats the number in scientific notation with uncertainty, producing \num{1.23(4)e5}.

	\paragraph*{Display options}
	Many different display options for the units and numeric notations can be configured in \verb|\sisetup{...}| globally or locally for each command.

	The \verb|\num| command can be forced into scientific notation using the \\
	\verb|scientific-notation=true| option:\\
	- \verb|\num[exponent-mode=scientific]{000.000123}| produces \num[exponent-mode=scientific]{000.000123}.

	The \verb|\si| command can also format units as fractions:  \\
	- \verb|\si[per-mode=fraction]{\kg\per\m\per\s\squared}| produces \si[per-mode=fraction]{\kg\per\m\per\s\squared}.\\
	- \verb|\si[per-mode=symbol]{\kg\per\m\per\s\squared}| produces \si[per-mode=symbol]{\kg\per\m\per\s\squared}.

	- The \verb|\SI[separate-uncertainty=false]{1.23(0.05)}{\metre}| command combines uncertainty with units and displays it separately, producing \SI[separate-uncertainty=false]{1.23(0.05)}{\metre}.







$\set{10}$

$\set[10]$

$\set(10) \qquad \qty(10)$


\begin{equation}
	\pdv{\psi}{t}
	=
	\frac{\partial \psi}{\partial t}
\end{equation}

