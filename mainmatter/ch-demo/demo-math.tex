\section{Mathematics and Units}

\subsection{Greek Symbols in Text}

% Use the `textgreek` package for inline Greek characters without entering math mode.
With the \verb|textgreek| package, you can use commands like \verb|\textalpha| (\textalpha) to create Greek characters inside normal text without needing to invoke math mode.

\subsection{Variables and Constants}

% Follow standard conventions: variables in italics, constants upright.
It is common practice to write variables in \textit{italics} and constants upright. For example, $e$ might be used for eccentricity, while $\mathrm{e}$ is used for Euler's constant $(=\num{2.718281}\dots)$.

% TODO: Add a table summarizing all math shortcuts.

\subsection{The \texttt{siunitx} Package}

% The `siunitx` package ensures correct typesetting of numbers and units.
This package is useful for automatic typesetting of numbers and units. You will need to choose appropriate settings in \verb|\sisetup{...}| in the \verb|main.tex| preamble. % TODO: Specify the exact settings path.

\paragraph*{Only Units}

% Use `\si` for proper spacing and formatting of standalone units.
The \verb|\si| command ensures proper spacing and styling for standalone units:
\begin{itemize}
    \item \verb|\si{\kilogram\metre\per\second}| produces \si{\kilogram\metre\per\second}.
    \item \verb|\si{\kg\per\m\per\s\squared}| produces \si{\kg\per\m\per\s\squared}.
    \item \verb|\si{\degreeCelsius}| correctly handles special symbols: \si{\degreeCelsius}.
    \item \verb|\si{\ohm}| ensures correct typesetting of units like ohms: \si{\ohm}.
\end{itemize}

\paragraph*{Only Numbers}

% Use `\num` to ensure consistent number formatting.
The \verb|\num| command formats numbers consistently:
\begin{itemize}
    \item \verb|\num{12345.678}| formats large numbers: \num{12345.678}.
    \item \verb|\num{6.022e23}| correctly formats scientific notation: \num{6.022e23}.
    \item \verb|\num{000.000123}| removes unnecessary leading zeros: \num{000.000123}.
\end{itemize}

\paragraph*{Numbers and Units}

% Use `\SI` to format numbers together with units.
The \verb|\SI| command combines numbers and units seamlessly:
\begin{itemize}
    \item \verb|\SI{9.81}{\metre\per\second\squared}| represents acceleration: \SI{9.81}{\metre\per\second\squared}.
    \item \verb|\SI{1.5}{\mega\hertz}| correctly applies SI prefixes: \SI{1.5}{\mega\hertz}.
    \item \verb|\SI{500}{\milli\liter}| handles subunits properly: \SI{500}{\milli\liter}.
\end{itemize}

\paragraph*{Uncertainty}

% Uncertainty notation in `siunitx`.
\begin{itemize}
    \item \verb|\num{1.23(45)}| shows uncertainty in parentheses: \num{1.23(45)}.
    \item \verb|\SI{1.23(45)}{\metre}| combines uncertainty with units: \SI{1.23(45)}{\metre}.
    \item \verb|\num{1.23(4)e5}| formats numbers with uncertainty in scientific notation: \num{1.23(4)e5}.
\end{itemize}

\paragraph*{Display Options}

% Customize `siunitx` display settings.
Different display options for units and numeric notation can be configured in \verb|\sisetup{...}| globally or locally.

The \verb|\num| command can be forced into scientific notation using the \verb|scientific-notation=true| option:
\begin{itemize}
    \item \verb|\num[exponent-mode=scientific]{000.000123}| produces \num[exponent-mode=scientific]{000.000123}.
\end{itemize}

The \verb|\si| command can format units as fractions:
\begin{itemize}
    \item \verb|\si[per-mode=fraction]{\kg\per\m\per\s\squared}| produces \si[per-mode=fraction]{\kg\per\m\per\s\squared}.
    \item \verb|\si[per-mode=symbol]{\kg\per\m\per\s\squared}| produces \si[per-mode=symbol]{\kg\per\m\per\s\squared}.
\end{itemize}

To display uncertainty separately, use:
\begin{itemize}
    \item \verb|\SI[separate-uncertainty=false]{1.23(0.05)}{\metre}|, which produces \SI[separate-uncertainty=false]{1.23(0.05)}{\metre}.
\end{itemize}







$\set{10}$

$\set[10]$

$\set(10) \qquad \qty(10)$


\begin{equation}
	\pdv{\psi}{t}
	=
	\frac{\partial \psi}{\partial t}
\end{equation}

