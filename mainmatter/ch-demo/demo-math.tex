\section{Mathematics and Units}


\begingroup \small
\subsection{Custom commands}

\I

\subsection{Physics package}

The \verb|physics| package defines \verb|\qty|. However this is already used by \texttt{siunitx}. Therefore this template redefines \verb|\qty| to set \verb|\set|. I can only be used in a math environment.


\verb|\set{10}| produces: $\set{10}$

\verb|\set[\dfrac{13}{14}]|     produces:$\set[\dfrac{13}{14}]$

$\set10 $

\subsubsection{Mathematical fonts}

mathds $\mathds{R, I, N, 1}$

mathscr $\mathscr{R, I, N, 1, 2, e, j, i, a}$

mathcal $\mathcal{R, I, N, 1, 2, e,h, j, i, a}$

mathfrak $\mathfrak{R, I, N, 1, 2, e, j, i, a}$

mathsf $\mathsf{R, I, N, 1, 2, e, j, i, a}$


\subsubsection{Differentials}

The \texttt{physics} packages adds various helpful commands for easy and quick setting of differentials.

The normal upright d used in by \texttt{physics} was replaced by $\mathrm{d\!I}$ which is used by professor Schiefermayr in his scripts. You can disable this in the \texttt{preamble/math.tex} under the custom math commands.

 $\dd[3]{x}$

\subsection{Greek Symbols in Text}

With the \verb|textgreek| package, you can use commands like \verb|\textalpha| (\textalpha) to create Greek characters inside normal text without needing to invoke math mode.

\subsection{Variables and Constants}

It is common practice to write variables in \textit{italics} and constants upright. For example, $e$ might be used for eccentricity, while $\mathrm{e}$ is used for Euler's constant $(=\num{2.718281}\dots)$.

% TODO: Add a table summarizing all math shortcuts.

\subsection{The \texttt{siunitx} Package}

% The `siunitx` package ensures correct typesetting of numbers and units.
This package is useful for automatic typesetting of numbers and units. You will need to choose appropriate settings in \verb|\unitsetup{...}| in the \verb|main.tex| preamble. % TODO: Specify the exact settings path.

\paragraph*{Only Units}

% Use `\unit` for proper spacing and formatting of standalone units.
The \verb|\unit| command ensures proper spacing and styling for standalone units:
\begin{itemize}
    \item \verb|\unit{\kilogram\metre\per\second}| produces \unit{\kilogram\metre\per\second}.
    \item \verb|\unit{\kg\per\m\per\s\squared}| produces \unit{\kg\per\m\per\s\squared}.
    \item \verb|\unit{\degreeCelsius}| correctly handles special symbols: \unit{\degreeCelsius}.
    \item \verb|\unit{\ohm}| ensures correct typesetting of units like ohms: \unit{\ohm}.
\end{itemize}

\paragraph*{Only Numbers}

% Use `\num` to ensure consistent number formatting.
The \verb|\num| command formats numbers consistently:
\begin{itemize}
    \item \verb|\num{12345.678}| formats large numbers: \num{12345.678}.
    \item \verb|\num{6.022e23}| correctly formats scientific notation: \num{6.022e23}.
    \item \verb|\num{000.000123}| removes unnecessary leading zeros: \num{000.000123}.
\end{itemize}

\paragraph*{Numbers and Units}

% Use `\qty` to format numbers together with units.
The \verb|\qty| command combines numbers and units seamlessly:
\begin{itemize}
    \item \verb|\qty{9.81}{\metre\per\second\squared}| represents acceleration: \qty{9.81}{\metre\per\second\squared}.
    \item \verb|\qty{1.5}{\mega\hertz}| correctly applies SI prefixes: \qty{1.5}{\mega\hertz}.
    \item \verb|\qty{500}{\milli\liter}| handles subunits properly: \qty{500}{\milli\liter}.
\end{itemize}

\paragraph*{Uncertainty}

% Uncertainty notation in `siunitx`.
\begin{itemize}
    \item \verb|\num{1.23(45)}| shows uncertainty in parentheses: \num{1.23(45)}.
    \item \verb|\qty{1.23(45)}{\metre}| combines uncertainty with units: \qty{1.23(45)}{\metre}.
    \item \verb|\num{1.23(4)e5}| formats numbers with uncertainty in scientific notation: \num{1.23(4)e5}.
\end{itemize}
To display uncertainty separately, use:
\begin{itemize}
    \item \verb|\qty[separate-uncertainty=false]{1.23(0.05)}{\metre}|, \\
    which produces \qty[separate-uncertainty=false]{1.23(0.05)}{\metre}.
\end{itemize}


\paragraph*{Display Options}

% Customize `siunitx` display settings.
Different display options for units and numeric notation can be configured in \\
 \verb|\unitsetup{...}| globally or locally.

The \verb|\num| command can be forced into scientific notation using the\\
\verb|scientific-notation=true| option:

\begin{itemize}
    \item \verb|\num[exponent-mode=scientific]{000.000123}| produces \num[exponent-mode=scientific]{000.000123}.
\end{itemize}

The \verb|\unit| and \verb|\qty| command can format units as fractions:
\begin{itemize}
    \item \verb|\unit[per-mode=fraction]{\kg\per\m\per\s\squared}| produces \unit[per-mode=fraction]{\kg\per\m\per\s\squared}.
    \item \verb|\unit[per-mode=symbol]{\kg\per\m\per\s\squared}| produces \unit[per-mode=symbol]{\kg\per\m\per\s\squared}.
\end{itemize}


















\endgroup