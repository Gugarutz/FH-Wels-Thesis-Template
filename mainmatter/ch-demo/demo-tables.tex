\section{Insert a table}

\blindtext

\begin{table}[H]
	\centering
	\caption{Testtabelle 2}
	\begin{tabular}{c|c||c}
		Heading column 1 & Heading column 2 & Heading column 3 \\ \headrule
		       1         &      Test-1      &      Test-2      \\ \hline
		       2         &        0         &        0         \\
		       3         &        0         &        0         \\
		       4         &        0         &        1         \\
	\end{tabular}
	\label{tab:Testtabelle-2}
\end{table}

\blindtext

\begin{table}[H]
    \centering
    \caption{Table with alternating row colors}
    \rowcolors{2}{gray!20}{white}  % Alternating row colors (light gray and white)
    \begin{tabular}{|p{3cm}|p{3cm}|p{3cm}|}
        \hline
          size   &  size 1  &  size 2  \\ \midrule
          unit   & [unit 1] & [unit 2] \\ \hline
        object 1 &    5     &    6     \\
        object 2 &    8     &    9     \\
        object 3 &    11    &    12    \\ \hline
    \end{tabular}
\end{table}

The following table uses \texttt{tabularx}, a package for automatic column width. The compatibility with \texttt{rowcolors} is limited
\begin{table}[H]
    \centering
    \caption{Table with alternating colours and automatic column width (B) using tabularx, fixed size (C) and standard centred (A)}
    \rowcolors{2}{white}{gray!15}
    \begin{tabularx}{0.8\linewidth}{c|XX p{2cm}}
        \hline
        A  & B  & B  & C  \\ \hline
        1  & 2  & 2  & 3  \\
        4  & 5  & 5  & 6  \\
        7  & 8  & 8  & 9  \\
        10 & 11 & 11 & 12 \\ \hline
    \end{tabularx}
\end{table}

\texttt{tabulary} is an extended version of \texttt{tabularx} with support for Right, Centred, Left aligned and justified text. The compatibility with \texttt{rowcolors} is limited

\begin{table}[H]
    \centering
    \caption{Example using tabulary, smaller text to save space}
    \small
    \begin{tabulary}{\linewidth}{|L|C|R|J|}
        \hline
        Left-aligned & Centered & Right-aligned & Justified text \\ \hline
        Apple  & 10   & 5.99  & This is some longer text that will wrap properly. \\
        Banana & 20   & 3.49  & Another line with text to test justification. \\
        Cherry & 15   & 7.99  & Yet another example. \\ \hline
    \end{tabulary}
\end{table}







%------------------------------------------------------------------------
\section{Longtable and importing data}


The following uses a simple csv file.

\DTLloaddb{data2}{data/sample_data.csv}
\begin{table}[H]
    \centering
    \caption{Table from an imported CSV}
    \begin{tabular}{l*{3}{r}}
    \toprule
    \textbf{Col1} &\textbf{Col2} &\textbf{Col3} &\textbf{ColPct}
    \DTLforeach*{data2}{\1=Col1, \2=Col2, \3=Col3, \4=ColPct}
    {%
        \DTLiffirstrow{\\ \midrule}{\\}%
        \1 & \2 & \3 & \4 %
    }
    \\ \hline
    \end{tabular}
\end{table}

For the next two tables the same csv file is used. It is located in \texttt{./data/}




The table displayed in a specific format. The syntax for this is looks intimidating but is not that hard.

\DTLloaddb{data}{data/sales_data.csv}
\begingroup\small\rowcolors{2}{gray!15}{white}
\begin{longtable}{|l|*{4}{r|}}
    \caption{Longtable from a CSV file}\cr
    \hline
    \textbf{Date} & \textbf{Product} & \textbf{Units Sold} & \textbf{Price (€)} & \textbf{Total (€)} \\
    \midrule
    \endfirsthead

    \multicolumn{5}{c}{\textit{Continued from previous page(opt.)}} \\ \hline
    \textbf{Date} & \textbf{Product} & \textbf{Units Sold} & \textbf{Price (€)} & \textbf{Total (€)} \\
    \midrule
    \endhead

    \hline
    \multicolumn{5}{|r|}{\textit{(Continued on next page)}} \\
    \hline
    \endfoot

    \hline
    \endlastfoot

    % Data from your CSV file
    \DTLforeach*{data}{\1=Date, \2=Product, \3=Quantity Sold, \4=Price (€), \5=Total (€)}{%
        \DTLiffirstrow{}{\\} % No extra line before the first row
        \1 & \2 & \3 & \4 & \5 %
    }

\end{longtable}
\endgroup